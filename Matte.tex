\documentclass[a4paper,11pt]{article}
\usepackage[T1]{fontenc}
\usepackage[utf8]{inputenc}
\usepackage{lmodern}
\usepackage{amssymb}
\usepackage{amsmath}
\usepackage{mathtools}


\title{Matte}
\author{Jakob Tigerström}

\begin{document}
\maketitle
\tableofcontents
\newpage
\begin{flushleft}
\section{Föreläsning 1}
\subsection{Värdesiffror}
Ex1: Hur många vädresiffror har talen
\begin{enumerate}
  \item 251 3 st
  \item 0,251 3 st
  \item 0,001 1 st
  \item 250 2 eller 3 st\newline
$ 2,5*10^2 $ 2 st\newline
$ 2,50*10^2 $ 3 st
  \item 2500 2,3 eller 4 st
$ 2,5*10^3 $\newline
$ 2,50*10^3 $\newline
$ 2,500*10^3 $
  \item 250,0 4 st
\end{enumerate}
Multiplikation och division: Svara med lika många värdesiffror som det värde som har minst värdesiffror.\newline
5,22 *3.1 = 16,182 = 16.
\subsection{Addition och Subtraktion}
Minst antal decimaler avgör.\newline
$ 23,52 + 12,4 = 35,92  \approx 35,9 $\newline
$ 23,56 + 12,4 = 35,96  \approx 36,0 $
\subsection{Uppskatta storleksordning}
$ \frac{2,8*10^5}{3,2*10^3} $\newline
Storleksordningen på svaret är $ 10^2 $
\section{Föreläsning 2}
Omskrivning av formler\newline
Densitet: $ \rho = m/v $\newline
\paragraph{EX:1}
Beräkna densiteten för en sten som har volymen $ 12cm^3 $ och väger $ 36 g $.\newline
$ \rho = \frac{m}{v} = \frac{36}{12} = 3,0g/cm^3 $\newline \newline
\paragraph{EX:2}
Beräkna volymen av ett okänt föremål med densiteten $ 0,8g/cm^3 $ och väger $ 24g $.\newline
$ \rho = \frac{m}{v} $\newline
$ \rho * V \frac{m}{V} * V $\newline
$ \frac{\rho * V}{\rho} = m $\newline
$ V = \frac{m}{\rho} $\newline
$ V=m/\rho = 24/0,8 = 30cm3 $ \newline
Hooke lag\newline
$ F=k*\Delta l $\newline
F - kraft\newline
k - fjäderkonstant\newline
$ \Delta l $ - fjäderns förlägning\newline
\paragraph{EX:3}
Bestäm konstanten för en fjäder  som sträcks ut 18cm när den belastas med kraften 37N.\newline
$ F=k*\Delta l $\newline
$ \frac{F}{\Delta l} = k $\newline
$ k = \frac{F}{\Delta l} = \frac{37}{0,18} = 205,55... \approx 2,1*10^2 N/m $\newline
Formel för rörelse energi: $ w = \frac{mv^2}{2} $\newline
w - energi(J)\newline
m - massa(kg)\newline
h - höjd(m)\newline
g - gravitationskonstant.9,52m/s2\newline
v - hastighet(m/s)\newline
EX4:\newline
Beräkna rörelseenergin för en bil som väger 1200kg och kör 90km/h\newline
$ w = \frac{mv^2}{2} = \frac{1200*25^2}{2} = 375000 \approx 4*10^5 J = 400 kJ = 0,4 mJ $\newline
$ 90km = 90000m $\newline
$ 1h = 3600s $\newline
$ \frac{90000}{3600} = \frac{90}{3,6} = 25m/s $\newline
\newpage
\section{Föreläsning 3}
\subsection{Vektorer}
Storhet som har både \underline{storlek} och \underline{riktning}.\newline
Storheter där riktningen ej är relevant kallas \underline{skalärer}.\newline
\textbf{Att skriva vektorer:}\newline
\textbf{F}, (f)\newline
\textbf{Att rita vektorer:}\newline
$ \longrightarrow $\newline
Pilens riktning är vektorens riktning.\newline
Pilens längd är vektorens storlek.\newline
\textbf{Att addera två vektorer:}\newline
Parallellogrammetoden.\newline
Polygonmetoden\newline
Att multiplicera/dividera en vektor med en skalär(ett tal):\newline
Multiplicera vektorn v(med tak) med talet $ k, k>0 $.\newline
Sammar riktning ,storleken påverkas av $ k, k<0 $.\newline
Motsatta riktningen storleken påverkas av k.\newline
Komposanter(att dela upp en vektor)
$ (x1;y1)+(x2;y2) = (x1+x2;y1+y2) $
\newpage
\section{Föreläsning 4}
\subsection{Grundläggande algebra och prioriteringsregler}
När vi beräknar värdet av ett uttryck måste vi ta hänsyn tilll prioriterings reglerna.
\begin{enumerate}
  \item Paranteser
  \item Potenser
  \item Multiplikation och division
  \item Addition och division 
\end{enumerate}
\paragraph{EX:1}
$ \underbrace{20/4}_{\text{3}}\underbrace{+8-}_{\text{4}}\underbrace{6*2}_{\text{3}}=\underbrace{5+8}_{\text{3}}\underbrace{-12}_{\text{3}}=1 $\newline
\paragraph{EX:2}
$ \underbrace{2*}_{\text{3}}\underbrace{5^3}_{\text{2}}=\underbrace{2*125}_{\text{3}}=250 $
\paragraph{EX:3}
$\underbrace{(8+5)}_{\text{1}}\underbrace{^2}_{\text{2}}\underbrace{(16+14)}_{\text{1}}=\underbrace{13^2}_{\text{2}}\underbrace{*30}_{\text{3}}=\underbrace{169*30}_{\text{3}}=5070 $

Addition $ term+term=summa $
Subtraktion $ term-term=differens $
Multiplikation $ faktor*faktor=produkt $
Divistion $ \frac{täljare}{nämnare}=kvot $
\subsection{Bråkräkning}
Multiplikation $ \frac{3}{5}*\frac{8}{7}=\frac{24}{35} $\newline
Täljare multipliceras till en täljare.\newline
nämnare multipliceras till en nämnare.\newline
Addition och subtraktion.\newline
$ \frac{1}{3}+\frac{1}{8}=\frac{8*1}{8*3}+\frac{1*3}{8*3}=\frac{8}{24}+\frac{3}{24}=\frac{11}{24} $
\end{flushleft}
\end{document}
