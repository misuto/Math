\documentclass[a4paper,11pt]{article}
\usepackage[T1]{fontenc}
\usepackage[utf8]{inputenc}
\usepackage{lmodern}
\usepackage{amssymb}
\usepackage{amsmath}
\usepackage{mathtools}


\title{Matte}
\author{Jakob Tigerström/Eric Johansson}

\begin{document}
\maketitle
\tableofcontents
\newpage
\begin{flushleft}
\section{TODO}
\begin{enumerate}
  \item Skriv fler föreläsningar
  \item Kolla stavning
  \item Fixa warnings
\end{enumerate}
\section{Föreläsning 1}
\subsection{Värdesiffror}
Ex1: Hur många vädresiffror har talen
\begin{enumerate}
  \item 251 3 st
  \item 0,251 3 st
  \item 0,001 1 st
  \item 250 2 eller 3 st\newline
$ 2,5*10^2 $ 2 st\newline
$ 2,50*10^2 $ 3 st
  \item 2500 2,3 eller 4 st
$ 2,5*10^3 $\newline
$ 2,50*10^3 $\newline
$ 2,500*10^3 $
  \item 250,0 4 st
\end{enumerate}
Multiplikation och division: Svara med lika många värdesiffror som det värde som har minst värdesiffror.\newline
5,22 *3.1 = 16,182 = 16.
\subsection{Addition och Subtraktion}
Minst antal decimaler avgör.\newline
$ 23,52 + 12,4 = 35,92  \approx 35,9 $\newline
$ 23,56 + 12,4 = 35,96  \approx 36,0 $
\subsection{Uppskatta storleksordning}
$ \frac{2,8*10^5}{3,2*10^3} $\newline
Storleksordningen på svaret är $ 10^2 $
\section{Föreläsning 2}
Omskrivning av formler\newline
Densitet: $ \rho = m/v $\newline
\paragraph{EX:1}
Beräkna densiteten för en sten som har volymen $ 12cm^3 $ och väger $ 36 g $.\newline
$ \rho = \frac{m}{v} = \frac{36}{12} = 3,0g/cm^3 $\newline \newline
\paragraph{EX:2}
Beräkna volymen av ett okänt föremål med densiteten $ 0,8g/cm^3 $ och väger $ 24g $.\newline
$ \rho = \frac{m}{v} $\newline
$ \rho * V \frac{m}{V} * V $\newline
$ \frac{\rho * V}{\rho} = m $\newline
$ V = \frac{m}{\rho} $\newline
$ V=m/\rho = 24/0,8 = 30cm3 $ \newline
Hooke lag\newline
$ F=k*\Delta l $\newline
F - kraft\newline
k - fjäderkonstant\newline
$ \Delta l $ - fjäderns förlägning\newline
\paragraph{EX:3}
Bestäm konstanten för en fjäder  som sträcks ut 18cm när den belastas med kraften 37N.\newline
$ F=k*\Delta l $\newline
$ \frac{F}{\Delta l} = k $\newline
$ k = \frac{F}{\Delta l} = \frac{37}{0,18} = 205,55... \approx 2,1*10^2 N/m $\newline
Formel för rörelse energi: $ w = \frac{mv^2}{2} $\newline
w - energi(J)\newline
m - massa(kg)\newline
h - höjd(m)\newline
g - gravitationskonstant.9,52m/s2\newline
v - hastighet(m/s)\newline
EX4:\newline
Beräkna rörelseenergin för en bil som väger 1200kg och kör 90km/h\newline
$ w = \frac{mv^2}{2} = \frac{1200*25^2}{2} = 375000 \approx 4*10^5 J = 400 kJ = 0,4 mJ $\newline
$ 90km = 90000m $\newline
$ 1h = 3600s $\newline
$ \frac{90000}{3600} = \frac{90}{3,6} = 25m/s $\newline
\newpage
\section{Föreläsning 3}
\subsection{Vektorer}
Storhet som har både \underline{storlek} och \underline{riktning}.\newline
Storheter där riktningen ej är relevant kallas \underline{skalärer}.\newline
\textbf{Att skriva vektorer:}\newline
\textbf{F}, (f)\newline
\textbf{Att rita vektorer:}\newline
$ \longrightarrow $\newline
Pilens riktning är vektorens riktning.\newline
Pilens längd är vektorens storlek.\newline
\textbf{Att addera två vektorer:}\newline
Parallellogrammetoden.\newline
Polygonmetoden\newline
Att multiplicera/dividera en vektor med en skalär(ett tal):\newline
Multiplicera vektorn v(med tak) med talet $ k, k>0 $.\newline
Sammar riktning ,storleken påverkas av $ k, k<0 $.\newline
Motsatta riktningen storleken påverkas av k.\newline
Komposanter(att dela upp en vektor)
$ (x1;y1)+(x2;y2) = (x1+x2;y1+y2) $
\newpage
\section{Föreläsning 4}
\subsection{Grundläggande algebra och prioriteringsregler}
När vi beräknar värdet av ett uttryck måste vi ta hänsyn tilll prioriterings reglerna.
\begin{enumerate}
  \item Paranteser
  \item Potenser
  \item Multiplikation och division
  \item Addition och division 
\end{enumerate}
\paragraph{EX:1}
$ \underbrace{20/4}_{\text{3}}\underbrace{+8-}_{\text{4}}\underbrace{6*2}_{\text{3}}=\underbrace{5+8}_{\text{3}}\underbrace{-12}_{\text{3}}=1 $\newline
\paragraph{EX:2}
$ \underbrace{2*}_{\text{3}}\underbrace{5^3}_{\text{2}}=\underbrace{2*125}_{\text{3}}=250 $
\paragraph{EX:3}
$\underbrace{(8+5)}_{\text{1}}\underbrace{^2}_{\text{2}}\underbrace{(16+14)}_{\text{1}}=\underbrace{13^2}_{\text{2}}\underbrace{*30}_{\text{3}}=\underbrace{169*30}_{\text{3}}=5070 $

Addition $ term+term=summa $
Subtraktion $ term-term=differens $
Multiplikation $ faktor*faktor=produkt $
Divistion $ \frac{täljare}{nämnare}=kvot $
\subsection{Bråkräkning}
Multiplikation $ \frac{3}{5}*\frac{8}{7}=\frac{24}{35} $\newline
Täljare multipliceras till en täljare.\newline
nämnare multipliceras till en nämnare.\newline
Addition och subtraktion.\newline
$ \frac{1}{3}+\frac{1}{8}=\frac{8*1}{8*3}+\frac{1*3}{8*3}=\frac{8}{24}+\frac{3}{24}=\frac{11}{24} $

\newpage
\section{Föreläsning 5}
\subsection{Algebra - uppställning och förenkling}
\textbf{EX1}\newline
Emil hyr en bil. Dygnsavgiften är 250kr och milkostnaden är 8kr/mil.\newline
A) Hur mycket kostar det ifall Emil hyr bilen i ett dygn och kör 12 mil.\newline
$ \underbrace{250}_{\text{Dygnsavg.}} + \underbrace{8*12}_{\text{mil kost.}} = 250+96 = 346kr $
Svar: Det kostar honom 346kr \newline\newline
B) Hur mycket ska Emil betala om han hyr bilen i k dygn och kör x mil?\newline\newline
$ \underbrace{250k}_{\text{Dyngsavg.}} + \underbrace{8k}_{\text{mil kost.}}$ <- Algebraiskt uttryck\newline\newline


\textbf{EX2}\newline
Annika lånar 15000kr för att köpa bil. Hon får betala 3\% i ränta.\newline
A) Hur stor är hennes skuld efter 5år om hon ej har betalt tillbaka något.\newline
$ \underbrace{15000}_{\text{Lån}} + \underbrace{1,03^5}_{\text{Förändringsfaktor}} \approx 17389kr $ \newline
\textit{$^5 = antal år$} \newline
Svar: Hon är skylldig ca 17389kr och är fast i lyxfällan \newline\newline
B) Hur stor är skulden efter x år?\newline\newline
$ \underbrace{15000}_{\text{Lån}} + \underbrace{1,03^x}_{\text{Förändringsfaktor}} $\newline\newline

\textbf{EX3}\newline
Förenkla: $4x+3x+6-2$.\newline\newline
$ \underbrace{4x+3x}_{\text{Addera}} + \underbrace{6-2}_{\text{subtrahera}} = 7x+4 $ \newline\newline

\textbf{EX4}\newline
Förenkla: $\frac{5}{4}a-\frac{a}{2}$.\newline\newline
$ \frac{5}{4}a-\underbrace{\frac{1}{2}a}_{\text{$\frac{a}{2}$}} = \frac{5}{4}a-\underbrace{\frac{1*2}{2*2}a}_{\text{Multiplicera}} = \frac{5}{4}a-\frac{2}{4}a = \frac{3}{4}a $\newline\newline

\textbf{EX5}\newline
Förenkla: $a(a+b)-b(a-7b)$.\newline\newline
$\underbrace{a(a+b)}_{\text{$a^2+ab$}} \underbrace{-}_{\text{-}} \underbrace{b(a+7b)}_{\text{$ab-7b^2$}} = a^2+ab-ab-7b^2 = a^2-7b^2 $\newline\newline


\newpage
\section{Föreläsning 7}
\subsection{Polynom}
Ett polynom är en summa av termer där variablernas exponenter är possitiva heltal.\newline
$ \underbrace{x^3+\overbrace{2}^{\text{Koeffcient}}x}_{\text{Variabel term}} - \underbrace{4}_{\text{Konstant term}} $
\subsection{Multiplicera polynom}
$ (a+b)(c+d) = ac+ad+bc+bd $\newline
$ (a+b+)(c+d+e) = ac+ad+ae+bc+bd+be $\newline
\subsection{Regler}
\subsubsection{Konjugat regeln}
$ \underbrace{(x+2)(x-2) }_{\text{Konjugat regeln}} = x^2-2x+2x-4 = x^2-4 $
\subsubsection{Kvadrerings regelerna}
$ \underbrace{(a+b)^2 = (a+b)(a+b) }_{\text{Kvadrerings regel}} = a^2+ab+ab+b^2 = a^2+2ab+b^2 $\newline
$ \underbrace{(a-b)^2 = (a-b)(a-b)}_{\text{Kvadrerings regel}} = a^2-ab-ab+b^2 = a^2-2ab+b^2 $\newline
\subsection{Uppgifter}
\subsubsection{EX1}
$ (a+5)(a-5) = a^2-5a+5a-25=a^2-25 $
\subsubsection{EX2}
$ (a+3)^2 = (a+3)(a+3) = a^2+6a+4 $
\subsubsection{EX3}
$ (3x+4y)^2 = 9x^2+2*3x*4y+16y^2 = 9x^2+24xy+16y^2 $
\subsubsection{EX4}
Faktorisera: $ 2xy^2+x^2y = xy(2y+x) $
\subsubsection{EX5}
Faktorisera: $ x^2-16 = (x+4)(x-4) $
\subsubsection{EX6}
Faktorisera: $ x^2+6x+9 = (x+3)^2 $
\subsubsection{EX7}
Faktorisera: $ 2x^2+10x+50 = 2(x^2+5x+25) $
\subsubsection{EX8}
Faktorisera: $ 5^x+5^{x+1} = 5^x+5^x*5 = 5^x(1+5 = 6*5^x $
\subsubsection{EX9}
Faktorisera: $ a^{2x+2}-a^{2x} = a^{2x}a^2-a^{2x} = a^{2x}(a^2-1) = a^{2x}(a+1)(a-1) $
\newpage
\section{Föreläsning 11}
\subsection{Logaritmer och logaritmlagar}
"Logaritmen av 2000 är det tal vi måste upphöja 10 med för att få 2000".\newline\newline
Definition: Om $ \underbrace{10^x = y}_{\text{potensform}} $ så är $ \underbrace{x=\log{y}}_{\text{logaritmform}} $\newline

Hur löser vi 10$^x$=1000? Detta är lätt att lösa, antingen vet man att $x=3$ eller så testar man olika värden på x tills man kommer till något i närheten.
Man kan även använda en grafritande räknare och kolla vart x skär 1000\newline
Hur löser vi 10$^x$=2000? Detta är ett mycket svårare tal att lösa och görs lättast genom att använda logaritm, men man kan även använda en grafritande räknare.\newline

$ \underbrace{10^x = 2000}_{\text{potensform}} $ -> $ \underbrace{x=\log{2000}}_{\text{logaritmform}} $\newline
Svaret blir: $ x \approx 3,301 $

\subsection{Logaritmlagarna}
$ a = 10^{\log{a}}$ \newline
Vi härleder logaritmlagarna med hjälp av potenslagarna
\subsubsection{1:a lagen}
AB = $10^{\log{A}}*10^{\log{B}} = 10^{\log{A}+\log{B}}$\newline
AB = $ 10^{\log{AB}} $ \newline
Lagen säger att "$\log{AB} = \log{A}+\log{B} $"
\subsubsection{2:a lagen}
$\frac{A}{B} = 10^{\log{A}}/10^{\log{B}} = 10^{(\log{A}-\log{B})}$\newline
$\frac{A}{B} = 10^{\log{A/B}} $ \newline
Lagen säger att "$\log{A/B} = \log{A}-\log{B} $"
\subsubsection{3:e lagen}
$A^k = \underbrace{A*A*A..*A}_{\text{k st}} = \underbrace{10^{\log{A}}*10^{\log{A}}*10^{\log{A}}..10^{\log{A}}}_{\text{k st}} = $\newline
$= (10^{\log{A}})^k = 10^{k*\log{A}}$\newline\newline
Lagen säger att "$\log(A^k) = k*\log{A} $"

\subsection{Logoritm exempel}
\textbf{EX1}\newline
Lös ekvationen $ 10^x = 67 $\newline\newline
$ \underbrace{10^x = 67}_{\text{potensform}} $ -> $ \underbrace{x=\log{67}}_{\text{logaritmform}} $\newline
Svaret blir: $ x \approx 1,8 $
\newline\newline
\textbf{EX2 - KONTROLLERA}\newline
Skriv talet 7 (exakt) som en potens med 10 som bas.\newline\newline
Svar: $ 7 = 10^{\log{7}} $
\newline\newline
\textbf{EX3}\newline
Lös ekvationen $ 2*\log{x}=12 $\newline\newline
$ 2*\log{x} = \underbrace{\frac{2*\log{x}}{2}}_{\text{Dividera med 2}} = \underbrace{\frac{12}{2}}_{\text{Dividera med 2}} = \log{x}=6 $ \newline\newline\newline
Svar: $ x = 10^6$
\newline\newline
\textbf{EX4 - FIXA}\newline
Lös exakt $ 3^x = 8 $\newline\newline
Alt1.\newline

Alt2.\newline

Svar: $ x = 1,9$
\newline\newline
\textbf{EX5}\newline
Lös: $\log{x} = \log{5}+\log{12}$\newline
Lösning med 1:a lagen.\newline\newline
$ \log{x} = \log{5}+\log{12} $\newline
$ \log{x} = \underbrace{\log{5*12}}_{\text{Gör om log12 till 12}} $ \newline\newline
$ \underbrace{\log{x}}_{\text{Ta bort log}} = \underbrace{\log{60}}_{\text{Ta bort log}}  $\newline\newline
$ x = 60$\newline
Svar: x = 60
\newline\newline
\textbf{EX6}\newline
Lös: $\log{x} = 2*\log{3}$\newline
Lösning med 3:e lagen. \newline\newline
$ \log{x} = 2*\log{3} $\newline
$ \log{x} = \log{3^2} $ \newline\newline
$ x = 3^2 $\newline\newline
Svar: x = 60
\newline\newline
\textbf{EX7}\newline
Lös: $\log{x^2} = 8$\newline
Lösning med 3:e lagen. \newline\newline
$ 2*\log{x} = 8 $\newline\newline
$ \underbrace{\frac{2*\log{x}}{2}}_{\text{Dividera med 2}} = \underbrace{\frac{8}{2}}_{\text{Dividera med 2}}  $ \newline\newline
$ \log{x} = 4 $\newline\newline
Svar: x = 4


\end{flushleft}
\end{document}
